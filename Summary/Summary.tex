\documentclass[11pt, twocolumn]{article}

    \usepackage{hyperref}
    
    \usepackage[T1]{fontenc}
    % Nicer default font (+ math font) than Computer Modern for most use cases
    \usepackage{mathpazo}

    % Basic figure setup, for now with no caption control since it's done
    % automatically by Pandoc (which extracts ![](path) syntax from Markdown).
    \usepackage{graphicx}
    % We will generate all images so they have a width \maxwidth. This means
    % that they will get their normal width if they fit onto the page, but
    % are scaled down if they would overflow the margins.
    \makeatletter
    \def\maxwidth{\ifdim\Gin@nat@width>\linewidth\linewidth
    \else\Gin@nat@width\fi}
    \makeatother
    \let\Oldincludegraphics\includegraphics
    % Set max figure width to be 80% of text width, for now hardcoded.
    \renewcommand{\includegraphics}[1]{\Oldincludegraphics[width=.8\maxwidth]{#1}}
    % Ensure that by default, figures have no caption (until we provide a
    % proper Figure object with a Caption API and a way to capture that
    % in the conversion process - todo).
    \usepackage{caption}

    \usepackage{adjustbox} % Used to constrain images to a maximum size 
    \usepackage{xcolor} % Allow colors to be defined
    \usepackage{enumerate} % Needed for markdown enumerations to work
    \usepackage{geometry} % Used to adjust the document margins
    \usepackage{amsmath} % Equations
    \usepackage{amssymb} % Equations
    \usepackage{textcomp} % defines textquotesingle
    % Hack from http://tex.stackexchange.com/a/47451/13684:
    \AtBeginDocument{%
        \def\PYZsq{\textquotesingle}% Upright quotes in Pygmentized code
    }
    \usepackage{upquote} % Upright quotes for verbatim code
    \usepackage{eurosym} % defines \euro
    \usepackage[mathletters]{ucs} % Extended unicode (utf-8) support
    \usepackage[utf8x]{inputenc} % Allow utf-8 characters in the tex document
    \usepackage{fancyvrb} % verbatim replacement that allows latex
    \usepackage{grffile} % extends the file name processing of package graphics 
                         % to support a larger range 
    % The hyperref package gives us a pdf with properly built
    % internal navigation ('pdf bookmarks' for the table of contents,
    % internal cross-reference links, web links for URLs, etc.)
    \usepackage{hyperref}
    \usepackage{longtable} % longtable support required by pandoc >1.10
    \usepackage{booktabs}  % table support for pandoc > 1.12.2
    \usepackage[inline]{enumitem} % IRkernel/repr support (it uses the enumerate* environment)
    \usepackage[normalem]{ulem} % ulem is needed to support strikethroughs (\sout)
                                % normalem makes italics be italics, not underlines
    \usepackage{mathrsfs}
    \usepackage{multicol}
    \usepackage{epsfig}
    \usepackage{graphicx}

    
    
    % Colors for the hyperref package
    \definecolor{urlcolor}{rgb}{0,.145,.698}
    \definecolor{linkcolor}{rgb}{.71,0.21,0.01}
    \definecolor{citecolor}{rgb}{.12,.54,.11}

    % ANSI colors
    \definecolor{ansi-black}{HTML}{3E424D}
    \definecolor{ansi-black-intense}{HTML}{282C36}
    \definecolor{ansi-red}{HTML}{E75C58}
    \definecolor{ansi-red-intense}{HTML}{B22B31}
    \definecolor{ansi-green}{HTML}{00A250}
    \definecolor{ansi-green-intense}{HTML}{007427}
    \definecolor{ansi-yellow}{HTML}{DDB62B}
    \definecolor{ansi-yellow-intense}{HTML}{B27D12}
    \definecolor{ansi-blue}{HTML}{208FFB}
    \definecolor{ansi-blue-intense}{HTML}{0065CA}
    \definecolor{ansi-magenta}{HTML}{D160C4}
    \definecolor{ansi-magenta-intense}{HTML}{A03196}
    \definecolor{ansi-cyan}{HTML}{60C6C8}
    \definecolor{ansi-cyan-intense}{HTML}{258F8F}
    \definecolor{ansi-white}{HTML}{C5C1B4}
    \definecolor{ansi-white-intense}{HTML}{A1A6B2}
    \definecolor{ansi-default-inverse-fg}{HTML}{FFFFFF}
    \definecolor{ansi-default-inverse-bg}{HTML}{000000}

    % commands and environments needed by pandoc snippets
    % extracted from the output of `pandoc -s`
    \providecommand{\tightlist}{%
      \setlength{\itemsep}{0pt}\setlength{\parskip}{0pt}}
    \DefineVerbatimEnvironment{Highlighting}{Verbatim}{commandchars=\\\{\}}
    % Add ',fontsize=\small' for more characters per line
    \newenvironment{Shaded}{}{}
    \newcommand{\KeywordTok}[1]{\textcolor[rgb]{0.00,0.44,0.13}{\textbf{{#1}}}}
    \newcommand{\DataTypeTok}[1]{\textcolor[rgb]{0.56,0.13,0.00}{{#1}}}
    \newcommand{\DecValTok}[1]{\textcolor[rgb]{0.25,0.63,0.44}{{#1}}}
    \newcommand{\BaseNTok}[1]{\textcolor[rgb]{0.25,0.63,0.44}{{#1}}}
    \newcommand{\FloatTok}[1]{\textcolor[rgb]{0.25,0.63,0.44}{{#1}}}
    \newcommand{\CharTok}[1]{\textcolor[rgb]{0.25,0.44,0.63}{{#1}}}
    \newcommand{\StringTok}[1]{\textcolor[rgb]{0.25,0.44,0.63}{{#1}}}
    \newcommand{\CommentTok}[1]{\textcolor[rgb]{0.38,0.63,0.69}{\textit{{#1}}}}
    \newcommand{\OtherTok}[1]{\textcolor[rgb]{0.00,0.44,0.13}{{#1}}}
    \newcommand{\AlertTok}[1]{\textcolor[rgb]{1.00,0.00,0.00}{\textbf{{#1}}}}
    \newcommand{\FunctionTok}[1]{\textcolor[rgb]{0.02,0.16,0.49}{{#1}}}
    \newcommand{\RegionMarkerTok}[1]{{#1}}
    \newcommand{\ErrorTok}[1]{\textcolor[rgb]{1.00,0.00,0.00}{\textbf{{#1}}}}
    \newcommand{\NormalTok}[1]{{#1}}
    
    % Additional commands for more recent versions of Pandoc
    \newcommand{\ConstantTok}[1]{\textcolor[rgb]{0.53,0.00,0.00}{{#1}}}
    \newcommand{\SpecialCharTok}[1]{\textcolor[rgb]{0.25,0.44,0.63}{{#1}}}
    \newcommand{\VerbatimStringTok}[1]{\textcolor[rgb]{0.25,0.44,0.63}{{#1}}}
    \newcommand{\SpecialStringTok}[1]{\textcolor[rgb]{0.73,0.40,0.53}{{#1}}}
    \newcommand{\ImportTok}[1]{{#1}}
    \newcommand{\DocumentationTok}[1]{\textcolor[rgb]{0.73,0.13,0.13}{\textit{{#1}}}}
    \newcommand{\AnnotationTok}[1]{\textcolor[rgb]{0.38,0.63,0.69}{\textbf{\textit{{#1}}}}}
    \newcommand{\CommentVarTok}[1]{\textcolor[rgb]{0.38,0.63,0.69}{\textbf{\textit{{#1}}}}}
    \newcommand{\VariableTok}[1]{\textcolor[rgb]{0.10,0.09,0.49}{{#1}}}
    \newcommand{\ControlFlowTok}[1]{\textcolor[rgb]{0.00,0.44,0.13}{\textbf{{#1}}}}
    \newcommand{\OperatorTok}[1]{\textcolor[rgb]{0.40,0.40,0.40}{{#1}}}
    \newcommand{\BuiltInTok}[1]{{#1}}
    \newcommand{\ExtensionTok}[1]{{#1}}
    \newcommand{\PreprocessorTok}[1]{\textcolor[rgb]{0.74,0.48,0.00}{{#1}}}
    \newcommand{\AttributeTok}[1]{\textcolor[rgb]{0.49,0.56,0.16}{{#1}}}
    \newcommand{\InformationTok}[1]{\textcolor[rgb]{0.38,0.63,0.69}{\textbf{\textit{{#1}}}}}
    \newcommand{\WarningTok}[1]{\textcolor[rgb]{0.38,0.63,0.69}{\textbf{\textit{{#1}}}}}
    
    
    % Define a nice break command that doesn't care if a line doesn't already
    % exist.
    \def\br{\hspace*{\fill} \\* }
    % Math Jax compatibility definitions
    \def\gt{>}
    \def\lt{<}
    \let\Oldtex\TeX
    \let\Oldlatex\LaTeX
    \renewcommand{\TeX}{\textrm{\Oldtex}}
    \renewcommand{\LaTeX}{\textrm{\Oldlatex}}
    % Document parameters
    % Document title
    \title{Group 4 Body Fat Prediction}
    \author{Chao Chang, Yezhou Li, Shuyang Chen, Ping Yu}
    
    
    
    
    

    % Pygments definitions
    
\makeatletter
\def\PY@reset{\let\PY@it=\relax \let\PY@bf=\relax%
    \let\PY@ul=\relax \let\PY@tc=\relax%
    \let\PY@bc=\relax \let\PY@ff=\relax}
\def\PY@tok#1{\csname PY@tok@#1\endcsname}
\def\PY@toks#1+{\ifx\relax#1\empty\else%
    \PY@tok{#1}\expandafter\PY@toks\fi}
\def\PY@do#1{\PY@bc{\PY@tc{\PY@ul{%
    \PY@it{\PY@bf{\PY@ff{#1}}}}}}}
\def\PY#1#2{\PY@reset\PY@toks#1+\relax+\PY@do{#2}}

\expandafter\def\csname PY@tok@w\endcsname{\def\PY@tc##1{\textcolor[rgb]{0.73,0.73,0.73}{##1}}}
\expandafter\def\csname PY@tok@c\endcsname{\let\PY@it=\textit\def\PY@tc##1{\textcolor[rgb]{0.25,0.50,0.50}{##1}}}
\expandafter\def\csname PY@tok@cp\endcsname{\def\PY@tc##1{\textcolor[rgb]{0.74,0.48,0.00}{##1}}}
\expandafter\def\csname PY@tok@k\endcsname{\let\PY@bf=\textbf\def\PY@tc##1{\textcolor[rgb]{0.00,0.50,0.00}{##1}}}
\expandafter\def\csname PY@tok@kp\endcsname{\def\PY@tc##1{\textcolor[rgb]{0.00,0.50,0.00}{##1}}}
\expandafter\def\csname PY@tok@kt\endcsname{\def\PY@tc##1{\textcolor[rgb]{0.69,0.00,0.25}{##1}}}
\expandafter\def\csname PY@tok@o\endcsname{\def\PY@tc##1{\textcolor[rgb]{0.40,0.40,0.40}{##1}}}
\expandafter\def\csname PY@tok@ow\endcsname{\let\PY@bf=\textbf\def\PY@tc##1{\textcolor[rgb]{0.67,0.13,1.00}{##1}}}
\expandafter\def\csname PY@tok@nb\endcsname{\def\PY@tc##1{\textcolor[rgb]{0.00,0.50,0.00}{##1}}}
\expandafter\def\csname PY@tok@nf\endcsname{\def\PY@tc##1{\textcolor[rgb]{0.00,0.00,1.00}{##1}}}
\expandafter\def\csname PY@tok@nc\endcsname{\let\PY@bf=\textbf\def\PY@tc##1{\textcolor[rgb]{0.00,0.00,1.00}{##1}}}
\expandafter\def\csname PY@tok@nn\endcsname{\let\PY@bf=\textbf\def\PY@tc##1{\textcolor[rgb]{0.00,0.00,1.00}{##1}}}
\expandafter\def\csname PY@tok@ne\endcsname{\let\PY@bf=\textbf\def\PY@tc##1{\textcolor[rgb]{0.82,0.25,0.23}{##1}}}
\expandafter\def\csname PY@tok@nv\endcsname{\def\PY@tc##1{\textcolor[rgb]{0.10,0.09,0.49}{##1}}}
\expandafter\def\csname PY@tok@no\endcsname{\def\PY@tc##1{\textcolor[rgb]{0.53,0.00,0.00}{##1}}}
\expandafter\def\csname PY@tok@nl\endcsname{\def\PY@tc##1{\textcolor[rgb]{0.63,0.63,0.00}{##1}}}
\expandafter\def\csname PY@tok@ni\endcsname{\let\PY@bf=\textbf\def\PY@tc##1{\textcolor[rgb]{0.60,0.60,0.60}{##1}}}
\expandafter\def\csname PY@tok@na\endcsname{\def\PY@tc##1{\textcolor[rgb]{0.49,0.56,0.16}{##1}}}
\expandafter\def\csname PY@tok@nt\endcsname{\let\PY@bf=\textbf\def\PY@tc##1{\textcolor[rgb]{0.00,0.50,0.00}{##1}}}
\expandafter\def\csname PY@tok@nd\endcsname{\def\PY@tc##1{\textcolor[rgb]{0.67,0.13,1.00}{##1}}}
\expandafter\def\csname PY@tok@s\endcsname{\def\PY@tc##1{\textcolor[rgb]{0.73,0.13,0.13}{##1}}}
\expandafter\def\csname PY@tok@sd\endcsname{\let\PY@it=\textit\def\PY@tc##1{\textcolor[rgb]{0.73,0.13,0.13}{##1}}}
\expandafter\def\csname PY@tok@si\endcsname{\let\PY@bf=\textbf\def\PY@tc##1{\textcolor[rgb]{0.73,0.40,0.53}{##1}}}
\expandafter\def\csname PY@tok@se\endcsname{\let\PY@bf=\textbf\def\PY@tc##1{\textcolor[rgb]{0.73,0.40,0.13}{##1}}}
\expandafter\def\csname PY@tok@sr\endcsname{\def\PY@tc##1{\textcolor[rgb]{0.73,0.40,0.53}{##1}}}
\expandafter\def\csname PY@tok@ss\endcsname{\def\PY@tc##1{\textcolor[rgb]{0.10,0.09,0.49}{##1}}}
\expandafter\def\csname PY@tok@sx\endcsname{\def\PY@tc##1{\textcolor[rgb]{0.00,0.50,0.00}{##1}}}
\expandafter\def\csname PY@tok@m\endcsname{\def\PY@tc##1{\textcolor[rgb]{0.40,0.40,0.40}{##1}}}
\expandafter\def\csname PY@tok@gh\endcsname{\let\PY@bf=\textbf\def\PY@tc##1{\textcolor[rgb]{0.00,0.00,0.50}{##1}}}
\expandafter\def\csname PY@tok@gu\endcsname{\let\PY@bf=\textbf\def\PY@tc##1{\textcolor[rgb]{0.50,0.00,0.50}{##1}}}
\expandafter\def\csname PY@tok@gd\endcsname{\def\PY@tc##1{\textcolor[rgb]{0.63,0.00,0.00}{##1}}}
\expandafter\def\csname PY@tok@gi\endcsname{\def\PY@tc##1{\textcolor[rgb]{0.00,0.63,0.00}{##1}}}
\expandafter\def\csname PY@tok@gr\endcsname{\def\PY@tc##1{\textcolor[rgb]{1.00,0.00,0.00}{##1}}}
\expandafter\def\csname PY@tok@ge\endcsname{\let\PY@it=\textit}
\expandafter\def\csname PY@tok@gs\endcsname{\let\PY@bf=\textbf}
\expandafter\def\csname PY@tok@gp\endcsname{\let\PY@bf=\textbf\def\PY@tc##1{\textcolor[rgb]{0.00,0.00,0.50}{##1}}}
\expandafter\def\csname PY@tok@go\endcsname{\def\PY@tc##1{\textcolor[rgb]{0.53,0.53,0.53}{##1}}}
\expandafter\def\csname PY@tok@gt\endcsname{\def\PY@tc##1{\textcolor[rgb]{0.00,0.27,0.87}{##1}}}
\expandafter\def\csname PY@tok@err\endcsname{\def\PY@bc##1{\setlength{\fboxsep}{0pt}\fcolorbox[rgb]{1.00,0.00,0.00}{1,1,1}{\strut ##1}}}
\expandafter\def\csname PY@tok@kc\endcsname{\let\PY@bf=\textbf\def\PY@tc##1{\textcolor[rgb]{0.00,0.50,0.00}{##1}}}
\expandafter\def\csname PY@tok@kd\endcsname{\let\PY@bf=\textbf\def\PY@tc##1{\textcolor[rgb]{0.00,0.50,0.00}{##1}}}
\expandafter\def\csname PY@tok@kn\endcsname{\let\PY@bf=\textbf\def\PY@tc##1{\textcolor[rgb]{0.00,0.50,0.00}{##1}}}
\expandafter\def\csname PY@tok@kr\endcsname{\let\PY@bf=\textbf\def\PY@tc##1{\textcolor[rgb]{0.00,0.50,0.00}{##1}}}
\expandafter\def\csname PY@tok@bp\endcsname{\def\PY@tc##1{\textcolor[rgb]{0.00,0.50,0.00}{##1}}}
\expandafter\def\csname PY@tok@fm\endcsname{\def\PY@tc##1{\textcolor[rgb]{0.00,0.00,1.00}{##1}}}
\expandafter\def\csname PY@tok@vc\endcsname{\def\PY@tc##1{\textcolor[rgb]{0.10,0.09,0.49}{##1}}}
\expandafter\def\csname PY@tok@vg\endcsname{\def\PY@tc##1{\textcolor[rgb]{0.10,0.09,0.49}{##1}}}
\expandafter\def\csname PY@tok@vi\endcsname{\def\PY@tc##1{\textcolor[rgb]{0.10,0.09,0.49}{##1}}}
\expandafter\def\csname PY@tok@vm\endcsname{\def\PY@tc##1{\textcolor[rgb]{0.10,0.09,0.49}{##1}}}
\expandafter\def\csname PY@tok@sa\endcsname{\def\PY@tc##1{\textcolor[rgb]{0.73,0.13,0.13}{##1}}}
\expandafter\def\csname PY@tok@sb\endcsname{\def\PY@tc##1{\textcolor[rgb]{0.73,0.13,0.13}{##1}}}
\expandafter\def\csname PY@tok@sc\endcsname{\def\PY@tc##1{\textcolor[rgb]{0.73,0.13,0.13}{##1}}}
\expandafter\def\csname PY@tok@dl\endcsname{\def\PY@tc##1{\textcolor[rgb]{0.73,0.13,0.13}{##1}}}
\expandafter\def\csname PY@tok@s2\endcsname{\def\PY@tc##1{\textcolor[rgb]{0.73,0.13,0.13}{##1}}}
\expandafter\def\csname PY@tok@sh\endcsname{\def\PY@tc##1{\textcolor[rgb]{0.73,0.13,0.13}{##1}}}
\expandafter\def\csname PY@tok@s1\endcsname{\def\PY@tc##1{\textcolor[rgb]{0.73,0.13,0.13}{##1}}}
\expandafter\def\csname PY@tok@mb\endcsname{\def\PY@tc##1{\textcolor[rgb]{0.40,0.40,0.40}{##1}}}
\expandafter\def\csname PY@tok@mf\endcsname{\def\PY@tc##1{\textcolor[rgb]{0.40,0.40,0.40}{##1}}}
\expandafter\def\csname PY@tok@mh\endcsname{\def\PY@tc##1{\textcolor[rgb]{0.40,0.40,0.40}{##1}}}
\expandafter\def\csname PY@tok@mi\endcsname{\def\PY@tc##1{\textcolor[rgb]{0.40,0.40,0.40}{##1}}}
\expandafter\def\csname PY@tok@il\endcsname{\def\PY@tc##1{\textcolor[rgb]{0.40,0.40,0.40}{##1}}}
\expandafter\def\csname PY@tok@mo\endcsname{\def\PY@tc##1{\textcolor[rgb]{0.40,0.40,0.40}{##1}}}
\expandafter\def\csname PY@tok@ch\endcsname{\let\PY@it=\textit\def\PY@tc##1{\textcolor[rgb]{0.25,0.50,0.50}{##1}}}
\expandafter\def\csname PY@tok@cm\endcsname{\let\PY@it=\textit\def\PY@tc##1{\textcolor[rgb]{0.25,0.50,0.50}{##1}}}
\expandafter\def\csname PY@tok@cpf\endcsname{\let\PY@it=\textit\def\PY@tc##1{\textcolor[rgb]{0.25,0.50,0.50}{##1}}}
\expandafter\def\csname PY@tok@c1\endcsname{\let\PY@it=\textit\def\PY@tc##1{\textcolor[rgb]{0.25,0.50,0.50}{##1}}}
\expandafter\def\csname PY@tok@cs\endcsname{\let\PY@it=\textit\def\PY@tc##1{\textcolor[rgb]{0.25,0.50,0.50}{##1}}}

\def\PYZbs{\char`\\}
\def\PYZus{\char`\_}
\def\PYZob{\char`\{}
\def\PYZcb{\char`\}}
\def\PYZca{\char`\^}
\def\PYZam{\char`\&}
\def\PYZlt{\char`\<}
\def\PYZgt{\char`\>}
\def\PYZsh{\char`\#}
\def\PYZpc{\char`\%}
\def\PYZdl{\char`\$}
\def\PYZhy{\char`\-}
\def\PYZsq{\char`\'}
\def\PYZdq{\char`\"}
\def\PYZti{\char`\~}
% for compatibility with earlier versions
\def\PYZat{@}
\def\PYZlb{[}
\def\PYZrb{]}
\makeatother


    % Exact colors from NB
    \definecolor{incolor}{rgb}{0.0, 0.0, 0.5}
    \definecolor{outcolor}{rgb}{0.545, 0.0, 0.0}



    
    % Prevent overflowing lines due to hard-to-break entities
    \sloppy 
    % Setup hyperref package
    \hypersetup{
      breaklinks=true,  % so long urls are correctly broken across lines
      colorlinks=true,
      urlcolor=urlcolor,
      linkcolor=linkcolor,
      citecolor=citecolor,
      }
    % Slightly bigger margins than the latex defaults
    
    \geometry{verbose,tmargin=1in,bmargin=1in,lmargin=1in,rmargin=1in}
    
    

    \begin{document}
    
    
    \maketitle


\noindent

\begin{abstract}
The current way of measuring body fat is complicated and inaccurate. Our group try to come up with a precise, robust and simple model with \textbf{two measurements} to calculate body fat.
\end{abstract}



\section{Data Cleaning}


In this section, we check some duplicate records, delete and impute some possible bad points. In the end, we delete $163rd,\ 182nd,\ 216th,\ 221st$ examples.
\\

From the histogram of BODYFAT, the $216th$ example has a body fat of 45.1\% and a density of 0.995, which is impossible for human. the $182nd$ example has a body fat of 0 and we get a negative body fat value using the body fat-density formula. Thus, these two point are deleted from the raw data set. 
\\

The histogram of HEIGHT shows that the $42rd$ example is only 29 inches, which is obviously a mistaken record. We impute his height using the BMI (adiposity) formula and the result is 69.4 inches.
\\

The $39th$ example has the largest value in weight, abdomen, chest, hip, thigh, knee, biceps, wrist and adiposity. We believe this man is just too fat and it may contains some useful information. We keep this point.
\\

We compute the ADIPOSITY  using weight and height to see whether there are some wrong records in this variable. The $163rd$ and $221st$ example has much larger difference between the records and computed values than other points. Thus, we believe the two points are mistaken records and delete them from raw data.

\section{Model Selection}
We compare the performance of two kinds of models: linear model and fraction model.

\subsection{Linear model}
We decide to use forward, backward selection with AIC and BIC as the criterion to choose the best linear model. The models given by AIC have more than eight variables and they are too complicated. Meanwhile, stepwise selection using BIC, no matter forward, backfard or in both direction give us the same model with four variables:
\begin{equation*}
\text{Bodyfat} \sim \text{Abdomen} + \text{Weight} + \text{Wrist} + \text{Forearm}
\end{equation*}

Next, we look into the multicollinearity issues among the four variables by checking their VIF values. In this step, we divide the data set into training set (70\%) and test set (30\%). The random seed here is $123$. Table \ref{vif} shows the result and it shows that from the full model, weight has the highest VIF value and it may have some multicollinearity problem. 
\\

In order to find two variables that has the best prediction power while suffer less from multicolliearity problem, we are going to drop the variables gradually to see whether the model performance changes a lot. Table \ref{linear.performance} shows the result. \textbf{ABDOMEN} is an important variable since by dropping it the $R^2$ drops to 0.17 and the RMSE increases. By making trade-off between RMSE and $R^2$, we conclude that the model with \textbf{ABDOMEN} and \textbf{WRIST} is the best linear model.
\begin{align*}
\text{Bodyfat} = 0.682 \times \text{Abeomen}\\
			    -2.022 \times \text{Wrist} -7.256
\end{align*}

\begin{table}
\centering
\begin{tabular}{l | c c c c }
\hline
Var & Weight & Abdomen & Forearm & Wrist \\
\hline 
VIF & 7.70 & 5.51 & 1.70 & 2.12 \\ 
\hline
\end{tabular}
\caption{VIF of four predictors}
\label{vif}
\end{table}

 \begin{table}[]
    \centering
    \begin{tabular}{l|ccc}
        \hline
         Models & RMSE & $R^2$ \\
         \hline
         Four variables & 4.17 & 0.71 \\
         Abdomen + Forearm + Wrist & 4.33 & 0.69 \\
         \textbf{Forearm + Wrist} & \textbf{7.08} & \textbf{0.17} \\
         Abdomen + Wrist & 4.33 & 0.69 \\
         Abdomen + Forearm & 4.34 & 0.68 \\
         Weight + Abdomen & 4.27 & 0.70 \\
         \hline
    \end{tabular}
    \caption{Model performance}
    \label{linear.performance}
\end{table}



\subsection{Fraction Model}
Given Siri equation, we notice that there is a linear relationship between \textbf{BodyFat} and $1 / \textbf{Density}$. By physical theorem, we know that \textbf{Density} can be calculated by \textbf{Volume} and \textbf{Weight}. Since we already possess accurate measure of \textbf{Weight}, the idea is to estimate volume using the remaining variables. Taking "Rule of Thumb" into account, we develop a new model named \textbf{Fraction Model}: 

$$\text{Bodyfat} = \beta_1 \frac{X_1}{ \text{Weight} }
				 + \beta_2 \frac{1}{ \text{Weight} } + \beta_3$$

We will use \textbf{10-fold Cross Validation} to find the best $X_i$ among the remaining 13 variables. The result of Cross Validation is shown in Table \ref{frac.cv}, increasingly sorted by \textbf{RMSE}. \\

As shown in Table \ref{frac.cv}, \textbf{ABDOMEN} ranks top in the table no matter sorted by RMSE or $R^2$. Therefore, we choose \textbf{ABDOMEN} as our optimal $X_1$. \\

Then we investigate further into the fraction model associated \textbf{ABDOMEN}. The summary of this model indicates that \textbf{intercept} is \textbf{not} significant given its p value is 0.48, while the other 2 variables are both significant. Hence, we drop intercept, which produces a model of 0.96 $R^2$:

$$\text{Bodyfat}_i = 153 \times \frac{ABDOMEN}{ \text{Weight} }
				 - \frac{1.06 \times 10^4}{ \text{Weight} }$$

 \begin{table}[]
    \centering
    \begin{tabular}{l|ccc}
        \hline
         Variable & RMSE & $R^2$ \\
         \hline
         \textbf{ABDOMEN} & \textbf{4.06} & \textbf{0.72} \\
         ADIPOSITY & 5.10 & 0.55 \\
         CHEST & 5.41 & 0.51 \\
         AGE & 5.51 & 0.47 \\
         WRIST & 5.73 & 0.43 \\
         HEIGHT & 5.75 & 0.51 \\
         HIP & 5.76 & 0.43 \\
         THIGH & 5.84 & 0.40 \\
         NECK & 5.87 & 0.41 \\
         ANKLE & 5.88 & 0.41 \\
         KNEE & 5.90 & 0.40 \\
         FOREARM & 5.92 & 0.39 \\
         BICEPS & 5.92 & 0.39 \\
         \hline
    \end{tabular}
    \caption{Fraction Model Selection}
    \label{frac.cv}
\end{table}



\subsection{Model Comparison}

We will compare the linear model with the fraction model to determine our final model. Table \ref{model.comp} shows the comparison of these two models. Given that fraction model's full data $R^2$ is noticeably larger than its 10-fold CV $R^2$, we conclude that the Fraction model \textbf{overfitts} on full data. In spite of that, we still choose it as our final model, because for both criteria, RMSE and $R^2$, \textbf{Fraction} model behaves \textbf{better} than \textbf{Linear model} no matter judged by Cross Validation or full data. This suggests that on average Fraction model has better generalization ability than Linear model. \\

The above paragraph shows that Fraction model is better than Linear model. Before we jump to any conclusion, we should check whether Fraction model satisfies model assumptions of Linear Regression Model. Figure \ref{fig:FracMdAspCheck} shows that model satisfies Linearity, Homoscedasticity, Normality of error term in general. However, in the upper-left figure, we notice a \textbf{"bowed"} pattern of the red line, indicating that Fraction model makes systematic errors especially when it's predicting \textbf{low} Bodyfat.

\begin{table}[]
\centering
\begin{tabular}{l|cc|cc}
\hline
      & \multicolumn{2}{c}{RMSE} & \multicolumn{2}{c}{$R^2$} \\
\hline
model & CV & Full Data & CV & Full data \\
\hline
 Linear & 4.20 & 4.21 & 0.70 & 0.68\\
 Fraction &  4.03 & 4.04 & 0.72 & 0.96 \\
 \hline
\end{tabular}
\caption{Model Comparison}
\label{model.comp}
\end{table}


\begin{figure}[h!]
  \centering
  \includegraphics{dig.png}
  \caption{Fraction Model Assumptions Check}
  \label{fig:FracMdAspCheck}
\end{figure}


\section{Shinny Application}
Our body fat prediction shinny app address is \url{https://clarefrost.shinyapps.io/shiny_628_group4/}.\\
In this app, we set different unit for weight (kg & lbs) and abdomen (cm & inches). We also put the histogram of body fat in the app for users' reference.


\section{Conclcusion}
In conclusion, we select \textbf{Fraction model as the final model.} 

$$\text{Bodyfat}_i = 153 \times \frac{ABDOMEN}{ \text{Weight} }
				 - \frac{1.06 \times 10^4}{ \text{Weight} }$$
The rule of thumb here is:$153$ times the ratio between abdomen and weight, and then minus $1.06*10^4$ times the inverse of weight.\\
One example can be: if a person has a weight of $184.5$lbs and his abdomen circumstance is $86.4$inches. Then his body fat is $153 \times \frac{86.4}{184.5} - \frac{1.06\times 10^4}{184.5}=13.75\%$.\\
This model is good in the sense that: 
\begin{itemize}
	\item Simple model: Use only two accessible measurements, weight and abdomen. 
	\item Precise model: Relatively low RMSE and high $R^2$.
	\item Logical model: Utilize Siri equation and estimation of density to develop model, so that there is physical logics in the model. 
\end{itemize}

However, this model also the following flaws: 

\begin{itemize}
	\item Overfitts on full data. 
	\item Systematically less precise predicting low and high Bodyfat. 
	\item Restriction on the amount of variables leads to a crude estimation of volume. 
\end{itemize}




\section{Contribution}

\begin{itemize}
	\item Chao Chang: Linear model variable selection with BIC/AIC in both direction; Build the shiny App.
	\item Yezhou Li: Fraction model variable selection with cross validation; Report summary.
	\item Shuyang Chen: Linear model variable selection with backward/ forward BIC/AIC; Organize code, figure and data file.
	\item Ping Yu: Linear model variable selection with Mallow's Cp and $R^2$; Report summary; Presentation slides.
\end{itemize}

\end{document}